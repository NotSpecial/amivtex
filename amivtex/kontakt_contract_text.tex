% Make contract sections paragraphs
\renewcommand*{\thesection}{\S\ \arabic{section}}

% The introduction
\begin{center}
\amivfairtitle
\bigbreak
\begin{huge}
Vertrag
\end{huge}
\medbreak
zwischen
\bigbreak
\end{center}

\vspace{2cm}

% Contract partners
\begin{minipage}[t]{0.4\textwidth}
\textbf{Aussteller}
\bigbreak
\companyname\\
\companyaddress\\
\companycity\\
\companycountry
\end{minipage}%
\begin{minipage}[t]{0.2\textwidth}
und
\end{minipage}%
\begin{minipage}[t]{0.4\textwidth}
\textbf{Veranstalter}
\bigbreak
\amivname\\
\amivaddress\\
\amivpostal\ \amivcity
\end{minipage}

\vfill
% Start of contract
\section*{Präambel}

Der AMIV an der ETH organisiert jährlich eine zweitägige Kontaktmesse an der
ETH Zürich. Die Messe richtet sich an Studierende der Studiengänge
`Elektrotechnik und Informationstechnologie',
`Maschinenbau und Verfahrenstechnik' sowie
`Management, Technology and Economics' und hat das Ziel, diese mit den
teilnehmenden Firmen in Kontakt zu bringen.

\newpage
\section{Vertragsgegenstand}
\subsection{Standplatz}

\companyboothchoice\hfill\companyboothprice
\companyboothinfo

\subsection{Teilnahmetag(e)}

\companydatechoice

\subsection{Pakete und Inserate}

\companyextrachoice

\section{Mietpreis}

Der Preis für den jeweils gewählten Stand beinhaltet Nebenkosten wie Strom und Internet sowie ein kleines Angebot für Essen und Trinken nach Vorgabe des Veranstalters.

\section{Mehrwertsteuer}

Alle genannten Preise verstehen sich ohne Mehrwertsteuer. Der Veranstalter kann gegebenenfalls die gesetzliche Mehrwertsteuer zusätzlich geltend machen.

\section{Zahlungsbedingungen}

Der Rechnungsbetrag ist binnen dreissig Tagen nach Ausstellung der Rechnung zur Zahlung fällig.

\section{Rücktritt}
\begin{enumerate}
	\item Ein Rücktritt des Ausstellers hat den Einbehalt eines pauschalierten
		Schadenersatzes als Kostenbeitrag zur Folge und zwar
	\begin{enumerate}
		\item Bei einem Rücktritt bis zu 30 Tagen vor dem Messetermin:\\
			30\% des Rechnungsbetrages,
		\item Bei einem Rücktritt bis zu 20 Tagen vor dem Messetermin:\\
			50\% des Rechnungsbetrages,
		\item Bei einem Rücktritt bis zu 10 Tagen vor dem Messetermin:\\
			75\% des Rechnungsbetrages,
		\item Bei einem Rücktritt weniger als 10 Tage vor dem Messetermin:\\
			100\% des Rechnungsbetrages.
	\end{enumerate}

	\item Inserat- und Paketoptionen (\S 1.3) bleiben von einem Rücktritt unberührt.
\end{enumerate}

\section{Termine}

\begin{enumerate}
\item Das Messeführerprofil des Ausstellers ist dem Veranstalter pünktlich online zur Verfügung zu stellen; eine Verspätung hat die Nichterwähnung im Messeführer etc. zur Folge.
\item Der Aufbau des Standes muss in der Zeit von 08:30 bis 10:30 Uhr des jeweils gebuchten Messetages erfolgen.
\item Der Abbau hat jeweils bis spätestens 19:00 Uhr abgeschlossen zu sein.
\end{enumerate}

\section{Kontakte}

Die Organisation der Veranstaltung erfolgt durch:
\bigbreak
\begin{tabular}{l}  % Tabular for identical layout to info below
\textbf{\amivname}\\
\amivaddress\\
\amivcity\\
\end{tabular}
\bigbreak
\begin{tabular}{l l}
Organisation: & Kommission Kontakt\\
Präsident: & \amivkontaktpresident\\
E-Mail: & \amivemail\\
\end{tabular}
\bigbreak
\noindent Sonderwünsche und Fragen sind an die genannten Kontaktadressen zu richten.

\section{Datenspeicherung}

Der Aussteller erklärt sich damit einverstanden, dass der Veranstalter die im Rahmen der Geschäftsbeziehung bekannt gewordenen Daten speichert.
Der Veranstalter verpflichtet sich, die Daten ausschliesslich im Zusammenhang mit diesem Vertrag zu verwenden.


\section{Haftungsausschluss}

Der Veranstalter haftet nur für solche Schäden, die er oder seine Mitarbeiter durch vorsätzliches oder grob fahrlässiges Verhalten verursachen. Ausgeschlossen sind insbesondere höhere Gewalt und sonstige unvorhergesehene Ereignisse etc.


\section{Sprache}

Der gesamte Vertragstext ist in deutscher und englischer Sprache abgefasst, 
wobei beide Fassungen als verbindlich gelten. Für rechtliche Interpretation hat 
jedoch der deutsche Text Vorrang.


\section{Salvatorische Klausel}

Sollte eine Bestimmung dieses Vertrages unwirksam oder undurchführbar sein oder werden, wird dadurch die Wirksamkeit der übrigen Bestimmungen nicht berührt. Die Vertragspartner werden vielmehr zusammenwirken, um an die Stelle der unwirksamen und undurchführbaren Bestimmung eine rechtlich zulässige und wirksame oder eine durchführbare Bestimmung zu ersetzen, welche geeignet ist, den mit der unwirksamen und undurchführbaren Bestimmung beabsichtigen Erfolg zu erreichen.

\section{Allgemeines}
\begin{enumerate}
\item Es gilt Schweizer Recht
\item Gerichtsstand für beide Parteien ist Zürich
\end{enumerate}

% End of contract, add fields to sign
\vfill
\noindent
\begin{minipage}[t]{0.45\textwidth}
\amivcity, den \hrulefill

\bigskip\noindent
\amivname\\
vertreten durch

\vspace{4em}
\hrulefill

\amivkontaktpresident

Kommission Kontakt, Präsident
\end{minipage}%
\hspace{0.1\textwidth}
\begin{minipage}[t]{0.45\textwidth}
\makebox[8em]{\hrulefill}, den \hrulefill

\bigskip\noindent
\companyname\\
vertreten durch

\vspace{4em}
\hrulefill

\companycontact\vphantom{Ensure correct height.}\\

\end{minipage}
